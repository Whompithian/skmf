%
% ----- copyright and title pages
%
\Title{Semantic Knowledge Management Framework\\for Learning Organizations}
\Author{Brendan Sweeney}
\Year{2015}
\Program{Cyber Security Engineering}

\Chair{Brent Lagesse}{Assistant Professor}{School of STEM Computing and Software Systems Division}
\Signature{Mark Kochanski}
\Signature{Marc Dupuis}
\Signature{Joe McCarthy}

%\copyrightpage

\Degreetext{A report\\submitted in partial fulfillment of the\\requirements of the degree of}
\Degree{Master of Science in Cyber Security Engineering}
\textofCommittee{Project Committee:}

\titlepage

\setcounter{page}{-1}
\abstract{Text of abstract.}

%
% ----- contents, etc.
%
\tableofcontents
%\listoffigures
%\listoftables

%
% ----- glossary
%
\chapter*{Glossary}      % starred form omits the `chapter x'
\addcontentsline{toc}{chapter}{Glossary}
\thispagestyle{plain}

\begin{glossary}
\item[object] In RDF, .
\item[ORM] In RDF, .
\item[predicate] In RDF, .
\item[RDF] Resource Description Framework. A cornerstone of the Semantic Web, RDF allows for uniquely identifiable (through URIs) resources to be describes in terms of triples which form a connected graph of semantically rich information.
\item[RDFS] In RDF, .
\item[Semantic Web] In RDF, .
\item[SPARQL] SPARQL Protocol and RDF Query Language. Used to manipulate data stored in the RDF format.
\item[SPARQL endpoint] SPARQL Protocol and RDF Query Language. Used to manipulate data stored in the RDF format.
\item[subject] In RDF, .
\item[triple] In RDF, .
\item[triplestore] In RDF, .
\item[URI] In RDF, .
 
\end{glossary}
 
%
% ----- acknowledgments
%
\acknowledgments{% \vskip2pc
  % {\narrower\noindent
  The author wishes to express sincere appreciation to
  University of Washington, where he has had the opportunity
  to work with .
  % \par}
}

%
% ----- dedication
%
\dedication{\begin{center}in memory of my father, Dr. Frank Sweeney\end{center}}

%
% end of the preliminary pages
