%
% ----- copyright and title pages
%
\Title{Semantic Knowledge Management Framework\\for Learning Organizations}
\Author{Brendan Sweeney}
\Year{2015}
\Program{Cyber Security Engineering}

\Chair{Brent Lagesse}{Assistant Professor}{School of STEM Computing and Software Systems Division}
\Signature{Mark Kochanski}
\Signature{Marc Dupuis}
\Signature{Joe McCarthy}

\copyrightpage

\Degreetext{A report\\
            submitted in partial fulfillment of the\\
            requirements of the degree of}
\Degree{Master of Science in Cyber Security Engineering}
\textofCommittee{Project Committee:}

\titlepage

\setcounter{page}{-1}
\abstract{Knowledge management systems (KMS) allow organizations to capture, refine, and redistribute knowledge between members with minimal effort. Learning organizations, in particular, can benefit from such systems thanks to the capability these systems provide to transfer knowledge from individuals into a repository that is readily accessible to the entire group. While a complete knowledge management system incorporates a multitude of tools and resources, a major cornerstone of any modern KMS is a digital datastore in which information can be collected and correlated. In this project, Semantic Knowledge Management Framework, the author uses Semantic Web technologies to create a flexible foundation for the datastore component of a knowledge management system. This report begins with a brief comparison between relational database systems and graph-based information systems, including the author's reasons for choosing a graph-based approach built on RDF. Some similar applications are examined, both using relational databases and using Semantic Web frameworks. Next, some of the design decisions are discussed, along with the particular challenges of adapting programming objects to a dynamic RDF model and preparing SPARQL statements without a predefined schema. The project is demonstrated for use by a hypothetical computer security incident response team to track and manage assets, patches, policies, and team members. Finally, future goals and directions for the project are proposed.}

%
% ----- contents, etc.
%
\tableofcontents
\listoffigures
%\listoftables

%
% ----- glossary
%
\chapter*{Glossary}      % starred form omits the `chapter x'
\addcontentsline{toc}{chapter}{Glossary}
\thispagestyle{plain}

\begin{glossary}
\item[object] In RDF, .
\item[ORM] In RDF, .
\item[predicate] In RDF, .
\item[RDF] Resource Description Framework. A cornerstone of the Semantic Web, RDF allows for uniquely identifiable (through URIs) resources to be describes in terms of triples which form a connected graph of semantically rich information.
\item[RDFS] In RDF, .
\item[Semantic Web] In RDF, .
\item[SKMF] Semantic Knowledge Management Framework; the software project that is discussed in this paper.
\item[SPARQL] SPARQL Protocol and RDF Query Language. Used to manipulate data stored in the RDF format.
\item[SPARQL endpoint] SPARQL Protocol and RDF Query Language. Used to manipulate data stored in the RDF format.
\item[subject] In RDF, .
\item[triple] In RDF, .
\item[triplestore] In RDF, .
\item[URI] In RDF, .
 
\end{glossary}
 
%
% ----- acknowledgments
%
\acknowledgments{% \vskip2pc
  % {\narrower\noindent
  The author wishes to express sincere appreciation to
  University of Washington, where he has had the opportunity
  to work with .
  % \par}
}

%
% ----- dedication
%
\dedication{\begin{center}in memory of my father, Dr. Frank Sweeney\end{center}}

%
% end of the preliminary pages
