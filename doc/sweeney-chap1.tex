\chapter{Introduction}
 
The utility of a clean, professionally prepared thesis is well
documented%
\footnote{See, for example,
  W.~Shakespeare\cite{Hamlet} for a recent discussion.}
and, even if you never intend to actually print your thesis,
you still ought to format it as if that were your intention.
 
\TeX\ facilitates that. It is a flexible,
complete and professional typesetting system.
It will produce {\bf pdf} output as required by the Graduate School.

\section{The Purpose of This Sample Thesis}
 
This sample is both a demonstration of the quality and
propriety of a \LaTeX formatted thesis and  
documentation for its preparation.
It has made extensive use of a custom class file
developed specifically for this purpose
at the University of Washington.  Chapter~II discusses
\TeX\ and \LaTeX.
Chapter III describes the additional macros and functions
provided by the custom thesis class file.  Finally, Chapter IV hopes to tie things up.
 
It is 
impossible to predict all the formatting problems one will encounter
and there will be problems that are best handled
by a specialist.  
The Graduate School may be able to help you find help.
Some departments may also be able to provide \LaTeX\ assistance.
 
 
\section{Conventions and Notations}
 
In this thesis the typist
refers to the user of \LaTeX---the one who
makes formatting decisions and chooses the appropriate
formatting commands.
He or she will most often be the degree candidate.
 
This document deals with \LaTeX\ typesetting commands and their
functions.  Wherever possible the conventions used to display
text entered by the typist and the resulting formatted output
are the same as those used by the \TeX books.
Therefore, {\tt typewriter type} is used to indicate text
as typed by the computer
or entered by the typist.
It is quite the opposite of {\it italics,} which indicates
a category rather than exact text.  For example,
{\tt alpha} and {\tt beta} might each be an example of a {\it label}.
 
 
\section{Nota bene}
 
This sample thesis was produced by the \LaTeX\ document class it describes
and its format is consonant with the Graduate School's electronic dissertation guidelines,
as of November, 2014, at least.
However, use of this package does not guarantee acceptability
of a particular thesis.
