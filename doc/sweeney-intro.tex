\chapter{Introduction}

Knowledge management - the practice of capturing, refining, and redistributing knowledge - is an important activity for any organization. It is of particular importance to organizations that operate with a high degree of knowledge transfer between employees; learning organizations. What sets knowledge management apart from information management is that information can be merely collected and stored, while knowledge must be actionable [KMS]. In order for that to happen, a mechanism must exist to allow for information to be refined, consolidated, correlated, and accurately retrieved. To that end, knowledge management systems (KMS) are designed to provide organizations with a centralized means to collect, refine, and redistribute knowledge between members. One important component of any modern KMS is a readily accessible digital datastore that allows members of an organization to contribute and refine information, then retrieve it based on some set of criteria, such as filters or keyword matches. The purpose of Semantic Knowledge Management Framework (SKMF) is to provide the means to construct such a datastore using Semantic Web technologies.


\section{The Semantic Web}

The Semantic Web was revealed by Tim Burners-Lee in 1998 as a means of opening up the Internet in a machine readable manner that allows simple Web services to locate and correlate information on behalf of users. While much of the need for the Semantic Web has been recently diminished by advances in machine learning, it still provides some technologies with novel characteristics for knowledge management. The two Semantic Web technologies used by SKMF are RDF and SPARQL.


\subsection{RDF and Triplestores}

Resource Description Framework (RDF) is the language that describes Semantic Web information. The smallest viable unit in RDF is the triple, a statement consisting of the URI of some resource, the URI of a specific property of that resource, and either a third URI or a literal value to provide meaning to the property. In RDF statements, these are referred to as the subject, predicate, and object, respectively. A collection of RDF statements provides a description of various resources and their relationships that can be represented as a directed graph, with subjects and objects as the nodes and predicates as the vertices directed from subjects to objects. While it is possible to serialize RDF statements into a form that is more compact than a complete list of triples, the information is typically kept in a triplestore that is further optimized for storage and retrieval of the statements, similar to a relational database system.

Relational database systems allow developers to define data models that map well to the objects in object-oriented programming. It is straightforward to assign each column from a table to an attribute of one programming object and define some predetermined behaviors to perform on those attributes. Due to its graph-based nature, RDF does not provide such a convenient mapping. In fact, the shape of the information changes as triples are inserted into and deleted from the triplestore. SKMF uses RDF, however, because the nature of information that will be placed in the system cannot be known beforehand. Any attempt by a developer to devise a schema of tables is merely a guess of what will be useful to the customer and in what way it will be useful. While very good guesses are possible, there will always missing attributes that the customer desires and extra attributes that the customer never uses, limiting how customers can represent their data. Despite its limitations, RDF is flexible enough to allow the customer define a personalized schema as information is added and forged into knowledge.


\subsection{SPARQL and the Endpoint}

SPARQL Protocol and RDF Query Language (SPARQL) is the language used to communicate with a triplestore in order to query the data or to modify it.


\section{No Foreign Keys}

Yet, everything could be considered a foreign key.


\section{Federated Searches}

Include outside data sources right within your regular queries.
