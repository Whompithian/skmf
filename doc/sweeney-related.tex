\chapter{Related Work}
 
Semantic Web technologies have been around since the turn of the millennium and knowledge management came into its modern form nearly a decade earlier
\cite{japanesekm}.
It took only a short time for the Semantic Web to be viewed as a viable framework for knowledge management systems. Since then, numerous research articles and software applications have been released to demonstrate the merits of semantic knowledge management. This chapter briefly examines some of this work in two parts. Section
\ref{research}
examines some of the literature that has been released around applying the Semantic Web to knowledge management. Section
\ref{applications}
provides an overview of some of the software applications that have used the Semantic Web in a knowledge management capacity, with particular attention given to those that inspired SKMF.

 
\section{Semantic Knowledge Management in Literature}
\label{research}

In 2002, Fensel described the On-to-Knowledge
\cite{ontoknow}
system for parsing electronic documents and creating RDF metadata from their contents, based on ontologies defined by the user. The proposed system supports impressive knowledge gathering, management, and sharing features. It would allow organizations to subscribe to large information sets, automatically catalog them, and share those catalogs with partners or within the organization. While the project completed successfully, it is difficult to determine what became of the results. Most of the external components used in the project can still be found, but those specific to the project and the project, itself, do not appear to have any recent information available. The project's website, www.ontoknowledge.org, returns `403 Forbidden' when viewed today.

Jovanović et al
\cite{semanticlearning}
looked to the Semantic Web in 2007 to evaluate and improve the effectiveness of knowledge dissemination in online learning environments. The authors conclude that the semantic technologies are able to provide a richer learning experience for learners and gather more meaningful feedback for instructors than the static forms often found on e-learning platforms at the time.

SKMF owes much of its inspiration to the 2013 paper by Gyrard et al
\cite{ontosec},
which introduces STAC, a Semantic Web ontology for attacks on computing systems and their corresponding countermeasures. While this is a limited application of knowledge management, it provides a clear demonstration of the practicality of such systems. More recently, Shenbagam and Salini
\cite{vulncontol}
describe an extension of this type of ontology
\colorbox{yellow}{to}
create a system for predicting and classifying attacks.
\colorbox{yellow}{The authors report}
higher success rates in attack prevention than afforded by conventional methods.


\section{Semantic Knowledge Management in Practice}
\label{applications}

Open Calais
\cite{opencalais}
offers much of what was promised by On-to-Knowledge, allowing users to subscribe to datasets and have semantic links pulled from them. Ontorion
\cite{ontorion}
is a proprietary system that combines a noSQL database with an algorithm to perform reasoning on large ontologies. The PoolParty suite
\cite{poolparty}
offers tools for applying semantic metadata to enterprise information sources. Finally, the STAC application
\cite{stacweb}
is a live implementation of the STAC ontology with a simple Web interface. Whereas the majority of semantic knowledge management applications focus on processing large amounts of existing data, the STAC application focuses on quickly providing relevant information in a specific knowledge domain.
