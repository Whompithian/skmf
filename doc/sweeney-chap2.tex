\chapter{Related Work}
 
The \TeX\ formatting program is the creation of
Donald Knuth of Stanford University.
It has been implemented on nearly every general purpose computer and
produces exactly the same copy on all machines.
 
\section{What is it; why is it spelled that way; 
and what do
really long section titles look like in the text and in the
Table of Contents?}
 
\TeX\ is a formatter.  A document's format is controlled
by commands embedded in the text.  
\LaTeX\ is a special version of \TeX---preloaded
with a voluminous set of macros that simplify most
formatting tasks.
 
\TeX\ uses {\it control sequences} to control
the formatting of a document.  These control sequences are usually
words or groups of letters prefaced with the backslash character
({\tt\char'134}).
For example,
Figure \ref{start-2} shows the text that printed the beginning
of this chapter.  Note the control sequence \verb"\chapter" that
instructed \TeX\ to start a new chapter, print the title, and
make an entry in the table of contents.  It is an example
of a macro defined by the \LaTeX\ macro package.
The control sequence \verb"\TeX", which prints the word \TeX,
is a standard macro from the {\it\TeX book}.
The short control sequence \verb"\\" in the title instructed \TeX\ to
break the title line at that point.
This capability is an example of an extension to \LaTeX\
provided by the uwthesis document class.
 
\begin{figure}
\begin{demo}
\singlespace
\\chapter\{A Brief\\\\Description of \\TeX\}

The \\TeX\\ formatting program is the creation of
Donald Knuth of Stanford University.
\end{demo}
\label{start-2}
\caption{The beginning of the Chapter II text}
\end{figure}
 
Most of the time \TeX\ is simply building paragraphs from
text in your source files.  No control sequences are involved.
New paragraphs are indicated by a blank line in the
input file.
Hyphenation is performed automatically.
 
\section{\TeX books}
 
The primary reference for \LaTeX\ is Lamport's second edition
of the \textit{\LaTeX\ User's Guide}\cite{Lbook}.
It is easily read and should be sufficient for thesis formatting.
See also the \textsl{\LaTeX\ Companion}\cite{companion} for descriptions
of many add-on macro packages.

Although unnecessary for thesis writers, the \textsl{\TeX book}
is the primary reference for \TeX sperts worldwide.
 
\section{Mathematics}
 
The thesis class does not expand on \TeX's
or \LaTeX's
comprehensive treatment of mathematical equation printing.%
\label{c2note}\footnote{%
% a long footnote indeed.
 Although many \TeX-formatted documents contain no
 mathematics except the page numbers, it seems appropriate
 that this paper, which is in some sense about \TeX,
 ought to demonstrate an equation or two.  Here then, is a statement 
 of the {\it Nonsense Theorem}.
 
 \smallskip
 \def\RR{{\cal R\kern-.15em R}}
 {\narrower\hangindent\parindent Assume a universe $E$ and a symmetric function
  $\$$ defined on $E$, such that for each $\$^{yy}$ there exists a
  $\$^{\overline{yy}}$, where $\$^{yy} = \$^{\overline{yy}}$.
  For each element $i$ of $E$ define
  ${\cal S}(i)=\sum_i \$^{yy}+\$^{\overline{yy}}+0$.
  Then if $\RR$ is that subset of $E$ where $1+1=3$,
  for each $i$
  $$\lim_{\$\to\infty}\int {\cal S}di =
      \cases{0,&if $i\not\in\RR$;\cr
             \infty,&if $i\in\RR$.\cr}$$
  \par}} % end of the footnote
%
The {\it\TeX book}\cite{book}, {\it \LaTeX\ User's Guide}\cite{Lbook},
and {\it The \LaTeX\ Companion}\cite{companion}
thoroughly cover this topic.
 
 
\section{Languages other than English}
 
Most \LaTeX\ implementations at the University are tailored
for the English language.  However, \LaTeX\ will format many
other languages.  Unfortunately, this author has never been successful in 
learning more than a smattering of anything other than English.
Consult your department or the Tex Users Group.
\smallskip
\begin{center}
{\tt http://tug.org/},
\end{center}
\smallskip
for assistance with non-English formatting.

Unusual characters can be defined via the
font maker \hbox{\mffont METAFONT} (documented by Knuth\cite{Metafont}).
The definitions are not trivial.
Students who attempt to print a thesis with
custom fonts may soon proclaim,
 
% Note.  This is not the ideal way to print Greek
\medskip
\begin{center}
``$\mathaccent"7027\alpha\pi o\kern1pt\theta\alpha\nu\epsilon\hat\iota\nu$
\ $\theta\acute\epsilon\lambda\omega$.''
 
\end{center}
